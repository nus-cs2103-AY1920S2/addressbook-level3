\documentclass[11pt,british]{article}

\usepackage[a4paper,margin=2.5cm]{geometry}
\usepackage[utf8]{inputenc}
\usepackage{inconsolata}
\usepackage{MnSymbol}
\usepackage[
	minionint,
	lf,
	mathtabular,
	loosequotes,
	swash,
	opticals]{MinionPro}
\usepackage{amsmath}%, amsfonts, amssymb}
% \usepackage{wasysym}
\usepackage[useregional,calc]{datetime2}
\usepackage[super]{nth}
\input{glyphtounicode}
\pdfgentounicode=1
\pdfminorversion=7

\usepackage[
	arc-separator = \,,
	retain-explicit-plus,
	detect-weight=true,
	detect-family=true,
	separate-uncertainty=true,
	multi-part-units=brackets]{siunitx}
\usepackage[version=4]{mhchem}
\usepackage[ISO]{diffcoeff}
\usepackage{bm}
\usepackage{esvect}

\usepackage{enumitem}
\usepackage{parskip}
\usepackage{multicol}
\usepackage{titlesec}
\usepackage{microtype}
\usepackage{bigfoot}

\usepackage{tabularx}
\usepackage{booktabs}

%\usepackage{listings}
\usepackage[usenames,dvipsnames]{xcolor}
\usepackage{pgfplots,pgfplotstable}
\usepackage[skins,theorems]{tcolorbox}
\usepackage{graphicx}
\usepackage{epstopdf}
\usepackage[labelfont={small,bf},font={small}]{caption}
\usepackage{subcaption}
\usepackage{float}
\tcbset{shield externalize,highlight math style={enhanced,
  colframe=red,colback=white,arc=0pt,boxrule=1pt}}
\pgfplotsset{compat=newest}
\usetikzlibrary{
	shapes,shapes.misc,
	arrows,arrows.meta,
	calc,positioning,
	decorations.pathreplacing,decorations.markings,
	decorations.text,calligraphy,
	pgfplots.dateplot,
	optics,external,
	circuits.ee.IEC
}
\tikzset{
	on each segment/.style={
		decorate,
		decoration={
			show path construction,
			moveto code={},
			lineto code={
				\path [#1]
				(\tikzinputsegmentfirst) -- (\tikzinputsegmentlast);
			},
			curveto code={
				\path [#1] (\tikzinputsegmentfirst)
				.. controls
				(\tikzinputsegmentsupporta) and (\tikzinputsegmentsupportb)
				..
				(\tikzinputsegmentlast);
			},
			closepath code={
				\path [#1]
				(\tikzinputsegmentfirst) -- (\tikzinputsegmentlast);
			},
		},
	},
	mid arrow/.style={postaction={decorate,decoration={
				markings,
				mark=at position .5 with {\arrow[#1]{Stealth}}
	}}},
	>=Stealth
}
\newcommand*\circled[1]{\tikz[baseline=(char.base)]{%
		\node[shape=circle, draw, minimum size=1.25em, inner sep=0pt, thick] (char) {#1};}}
\tikzexternalize[prefix=figures/]
\captionsetup{width=0.6\textwidth}

\usepackage[nottoc,numbib]{tocbibind}
\usepackage[
	backend=biber,
	language=british,
    backref=true,
	style=verbose-ieee,
    bibstyle=numeric,
    citestyle=numeric,
    sorting=none
]{biblatex}

%\addbibresource{'hall_effect_report.bib'}

\DefineBibliographyStrings{english}{%
    backrefpage = {page},% originally "cited on page"
    backrefpages = {pages},% originally "cited on pages"
}
\DeclareCiteCommand{\supercite}[\mkbibsuperscript]
{\iffieldundef{prenote}
    {}
    {\BibliographyWarning{Ignoring prenote argument}}%
    \iffieldundef{postnote}
    {}
    {\BibliographyWarning{Ignoring postnote argument}}}
{\usebibmacro{citeindex}%
    \bibopenbracket\usebibmacro{cite}\bibclosebracket}
{\supercitedelim}
{}
\let\cite=\supercite

\usepackage{hyperref}
\usepackage[capitalise,noabbrev,nameinlink]{cleveref}
\crefdefaultlabelformat{#2\textbf{#1}#3}
\creflabelformat{equation}{#2\textbf{(#1)}#3}
\crefname{equation}{\textbf{Equation}}{\textbf{Equations}}
\Crefname{equation}{\textbf{Equation}}{\textbf{Equations}}
\crefname{figure}{\textbf{Figure}}{\textbf{Figures}}
\Crefname{figure}{\textbf{Figure}}{\textbf{Figures}}
\crefname{table}{\textbf{Table}}{\textbf{Tables}}
\Crefname{table}{\textbf{Table}}{\textbf{Tables}}
\crefname{appendix}{\textbf{Appendix}}{\textbf{Appendices}}
\Crefname{appendix}{\textbf{Appendix}}{\textbf{Appendices}}
\crefname{section}{\textbf{\S}}{\textbf{\S}}
\Crefname{section}{\textbf{\S}}{\textbf{\S}}
\crefname{algorithm}{\textbf{Algorithm}}{\textbf{Algorithms}}
\Crefname{algorithm}{\textbf{Algorithm}}{\textbf{Algorithms}}

\hypersetup{
	unicode 	 = true,
	colorlinks   = true,           %Colours links instead of ugly boxes
	urlcolor     = PineGreen,     %Colour for external hyperlinks
	linkcolor    = NavyBlue,         %Colour of internal links
	citecolor    = Magenta,   %Colour of citations
	linktocpage  = true
}

\DTMlangsetup*{ord=raise}

\DeclareMathOperator{\sinc}{sinc}

\setlength{\jot}{10pt}

\setlist[itemize]{left=0pt}
\setlist[enumerate,1]{left=0pt}

%\lstdefinestyle{code}{
%	backgroundcolor=\color{backcolour},
%	commentstyle=\color{codegreen},
%	keywordstyle=\color{magenta},
%	numberstyle=\tiny\color{codegray},
%	stringstyle=\color{codepurple},
%	basicstyle=\ttfamily\footnotesize,
%	breakatwhitespace=false,
%	breaklines=true,
%	captionpos=b,
%	keepspaces=true,
%	numbers=left,
%	numbersep=5pt,
%	showspaces=false,
%	showstringspaces=false,
%	showtabs=false,
%	tabsize=2
%}

%\lstset{style=code}


\title{\vspace{-2.5cm} \textbf{CS2103T} \\ Team Project: CookBuddy}
\author{Sharadh Rajaraman \\ \textbf{A0189906L}}
\begin{document}
\maketitle
\section{Overview}\label{overview}
\href{https://github.com/AY1920S2-CS2103T-W12-4/main}{CookBuddy} is a desktop recipe manager for students living in on-campus accommodation, who enjoy cooking, and are also familiar with the command line.

Users may interact with CookBuddy through a Command Line Interface (CLI), and it has a Graphical User Interface (GUI) created with JavaFX.

CookBuddy is written in Java, and has about \num{10000} lines of code.

\section{Summary of contributions}\label{summaryofcontributions}
\href{https://nus-cs2103-ay1920s2.github.io/tp-dashboard/#search=W12&sort=groupTitle&sortWithin=title&since=2020-02-14&timeframe=commit&mergegroup=false&groupSelect=groupByRepos&breakdown=false&tabOpen=true&tabType=authorship&tabAuthor=sharadhr&tabRepo=AY1920S2-CS2103T-W12-4\%2Fmain\%5Bmaster\%5D}{This RepoSense panel} displays this author's contributions to the project.

\subsection{Enhancements Implemented}
\subsubsection{Major Enhancements}

\paragraph{Added image reading, UI display, and processing (PR \href{https://github.com/AY1920S2-CS2103T-W12-4/main/pull/278}{\texttt{\#278}})}\label{imgcontrib}

\begin{itemize}
    \item \textbf{What it does:} Allows users to add images to their recipes, by specifying a relative or absolute
    filepath. Images are saved in a data folder local to CookBuddy, and are named with a
    UID (unique identifier). If an image cannot be found at the specified path, a placeholder is used instead.

    \item \textbf{Justification}: Users should be able to see what the final dish looks
    like, and an image goes a very long way in showing that. Furthermore, images are a prominent feature of most competing recipe managers, and we feel this is a key feature that allows CookBuddy to compete on a similar standing.

    \item \textbf{Highlights}:
    \begin{itemize}
    	\item Adding reading images required a firm understanding of file I/O, as well as understanding the various Java classes in \href{https://docs.oracle.com/en/java/javase/13/docs/api/java.base/java/nio/file/package-summary.html}{\texttt{java.nio.file}}.

    	Appropriate methods for easy extensibility were implemented in \href{https://github.com/AY1920S2-CS2103T-W12-4/main/blob/master/src/main/java/cookbuddy/commons/util/PhotographUtil.java}{\texttt{cookbuddy.commons.util.PhotographUtil}}, as well as multiple constructors in \href{https://github.com/AY1920S2-CS2103T-W12-4/main/blob/master/src/main/java/cookbuddy/model/recipe/attribute/Photograph.java}{\texttt{cookbuddy.model.recipe.attribute.Photograph}}.

    	\item Other file read/write methods were also implemented in \href{https://github.com/AY1920S2-CS2103T-W12-4/main/blob/master/src/main/java/cookbuddy/commons/util/FileUtil.java}{\texttt{cookbuddy.commons.util.FileUtil}}.

    	\item \texttt{PhotographUtil} employs the \href{https://en.wikipedia.org/wiki/Singleton_pattern}{singleton pattern}, to prevent initialisation errors.
    \end{itemize}
\end{itemize}

\subsubsection{Minor Contributions}
\paragraph{Revamped the User Interface (UI) of CookBuddy (PR \href{https://github.com/AY1920S2-CS2103T-W12-4/main/pull/179}{\texttt{\#179}}) }
\begin{itemize}
	\item A light theme was preferred by the team members
	\item \href{https://pixelduke.com/java-javafx-theme-jmetro/}{\texttt{JMetro}} was used to change the theme
	\item Appropriate classes and corresponding \texttt{.fxml} files were generated to produce the new UI.
	\item The panels of the UI were also changed, with a fixed scrolling side panel for the recipe list view, and a selected recipe overview panel.
\end{itemize}

\paragraph{Added Ingredient and Instruction List Attributes (PR \href{https://github.com/AY1920S2-CS2103T-W12-4/main/pull/125}{\texttt{\#125}}) }
\begin{itemize}
	\item Ingredients and instructions were originally stored as un-delimited \texttt{String}s
	\item Classes to represent \texttt{Ingredient}s, \texttt{Instruction}s and their containers were implemented
\end{itemize}

\paragraph{Minor UI and Typeface Changes to Guides (Commit \href{https://github.com/AY1920S2-CS2103T-W12-4/main/pull/319/commits/01a9faffe83f12997f9eba13a2b92e6f4322167d}{\texttt{01a9faf}}) }

\subsection{Documentation Contributions}
\subsubsection{User Guide}
\begin{itemize}
	\item Added various feature details in \href{https://ay1920s2-cs2103t-w12-4.github.io/main/UserGuide.html#features}{\S{} 4}
	\item Added data file information in \href{https://ay1920s2-cs2103t-w12-4.github.io/main/UserGuide.html#configuration-and-recipe-data-sharadh}{\S{} 5}.
\end{itemize}

\subsubsection{Developer Guide}
\begin{itemize}
	\item \href{https://ay1920s2-cs2103t-w12-4.github.io/main/DeveloperGuide.html#design}{\S{} 5}, \href{https://ay1920s2-cs2103t-w12-4.github.io/main/DeveloperGuide.html#image-management-done-by-sharadh-rajaraman}{\S{} 6.2} were added, detailing the overall architecture, and the implementation of image processing (see \cref{imgcontrib}).
	\item Future work was briefly detailed in \href{https://ay1920s2-cs2103t-w12-4.github.io/main/DeveloperGuide.html#Implementation-Future}{\S{} 6.12};
	\item Appendices \href{https://ay1920s2-cs2103t-w12-4.github.io/main/DeveloperGuide.html#product-scope-done-by-zain-alam-sharadh-rajaraman}{A} and \href{https://ay1920s2-cs2103t-w12-4.github.io/main/DeveloperGuide.html#user-stories-done-by-sharadh-rajaraman-and-adarsh-chugani}{B} were added; in the latter, the User Stories were placed in a neat table.
\end{itemize}

\subsection{Project Management and Teamwork}
\paragraph{Project Management}
\begin{itemize}
	\item Set up the organisation, repository, and initial branches for all group members
	\item Set up \texttt{Travis CI} and \texttt{AppVeyor} for initial continuous integration testing
\end{itemize}

\paragraph{Teamwork}
\begin{itemize}
	\item Assisted weaker members in setting up \texttt{git} and with iP tasks
	\item Set up Zoom meetings to facilitate discussion of project proposals and code reviews, as well as implementation details
\end{itemize}
\end{document}
